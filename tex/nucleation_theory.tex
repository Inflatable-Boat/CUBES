% Credit: TeXniCie A-eskwadraat

\documentclass[thesis]{subfiles}

\begin{document}

\section{Nucleation Theory}

In this section we will discuss 
Before we dive into the simulations, we'll first discuss a simple approach to crystal nucleation, and then how we make a simulation which gives us nucleation in a reasonable time frame, using umbrella sampling.

\subsection{Classical Nucleation Theory}\label{subsec:cnt}

\begin{comment} % this is now in introduction.tex
The simplest model of nucleation is well known, and goes by the name of classical nucleation theory (CNT). The theory applies when we have a system in a \emph{metastable} state. Heuristically it is as follows. The entire system of interest is in state $A$ with a chemical potential $\mu_A$, but under current conditions (due to e.g. pressure, density) the Gibbs free energy of the system would be lower in state $B$, with a chemical potential $\mu_B < \mu_A$. However, by changing a droplet in the system from state $A$ to state $B$, we create an interface with surface tension $\gamma > 0$ between state $A$ and state $B$. CNT asserts that a droplet is spherical (which makes sense, as this gives the most volume for the least area), so the energy change from a droplet at a radius $r$ will be	
\begin{equation}
	\Delta G(r) = 4\pi r^2 \gamma - \frac{4\pi r^3 \Delta\mu}{3},
\end{equation}
where $\Delta\mu = \mu_A - \mu_B > 0$. Naturally, if making the droplet larger makes this energy cost is greater, i.e. if $\frac{d\Delta G(r)}{dr} > 0$, %than the energy gain from changing the state, 
then the system will, in order to minimize the Gibbs free energy, make the droplet smaller. However, because $r^3$ grows faster than $r^2$, at sufficiently high $r$ the bulk energy gain will overtake the surface energy cost. This can be easily calculated by setting the derivative to zero, and we find the critical droplet radius
\begin{equation}
	r^* = \frac{2\gamma}{\Delta\mu}.
\end{equation}
This results in an energy barrier of 
\begin{equation}
%	\Delta G^* = 
	\Delta G (r^*) = \frac{16 \pi \gamma^3}{3 \Delta\mu^2}.
\end{equation}
\end{comment}
\subsubsection{Nucleation Rate}




\end{document}













