% Credit: TeXniCie A-eskwadraat

\documentclass[thesis]{subfiles}


\begin{document}

\part{Introduction}

In the past few decades, advancements in colloidal particle synthesis have made it possible to create a wide range of particles with different shapes and interactions\cite{glotzer2007anisotropy}. One of the fascinating properties of these nanoparticles is that, because they undergo Brownian motion in a solvent, they can self-assemble into all sorts of different phases, ranging from fluids to all kinds of beautiful solid structures\todo{Ref.}. This zoo of nanoparticles gives us a rich sea of phases and other structures to study, as being able to control the structure of matter at the nano level can have many applications such as \todo{references} light-weight but high-strength materials all the way to medical applications.\\

In order to understand the physics of these colloidal systems we can characterize the different phases these particles can exhibit, and analyze how these systems go from one phase to another. Furthermore, anisotropic particles also carry an orientation and as such, can display both translational and orientational order. In particular, we are interested in how these two types of order form and interact with each other in the crystallization process, which is the question we are trying to probe in this thesis.\\

To this end, we will study the nucleation of hard (slanted) cubes. Hard cubes form an excellent candidate for studying these types of order during the crystallization process for a few reasons. Firstly, the phase behaviour of hard (slanted) cubes has been studied in the past\cite{van2017phase,smallenburg2012vacancy}, and it has been shown that hard cubes go from a disordered liquid to a simple cubic crystal with both translational as well as orientational order. Therefore we can use the knowledge of the phase boundary to study the question of how the two types of order interact with each other directly, rather than first having to determine the possible crystal structures that could form. Secondly, hard cubes are very easy to model and therefore are also not too expensive to simulate. Lastly, colloidal cubes are already possible to make relatively easily, and out of many different materials.\todo{refs} This gives us hope that we can compare theory and simulations with experiments in the end.

\newpage



\end{document}

