% Credit: TeXniCie A-eskwadraat

\documentclass[thesis]{subfiles}


\begin{document}

\section{Introduction}

In the past few decades, advancements in colloidal particle synthesis have made it possible to create a huge range of differently shaped particles. One such shape, the topic of this thesis, is the cube. One of the fascinating properties of these nanoparticles is their ability to self-assemble into a crystal. In this thesis we will try to study the structure formation in cubes, and see how a continuous transformation of such a shape affects these processes. 

\subsection{Classical Nucleation Theory}\label{subsec:cnt}

The simplest model of nucleation is well known, and goes by the name of classical nucleation theory (CNT). The theory applies when we have a system in a \emph{metastable} state. Heuristically it is as follows. The entire system of interest is in state $A$ with a chemical potential $\mu_A$, but under current conditions (due to e.g. pressure, density) the Gibbs free energy of the system would be lower in state $B$, with a chemical potential $\mu_B < \mu_A$. However, by changing a droplet in the system from state $A$ to state $B$, we create an interface between state $A$ and state $B$ with surface tension $\gamma > 0$. CNT asserts that a droplet is spherical (which makes sense, as this gives the most volume for the least area), so the energy change from a droplet at a radius $r$ will be	
\begin{equation}
\Delta G(r) = 4\pi r^2 \gamma - \frac{4\pi r^3 \Delta\mu}{3},
\end{equation}
where $\Delta\mu = \mu_A - \mu_B > 0$. Naturally, if making the droplet larger makes this energy cost is greater, i.e. if $\frac{d\Delta G(r)}{dr} > 0$, %than the energy gain from changing the state, 
then the system will, in order to minimize the Gibbs free energy, make the droplet smaller. However, because $r^3$ grows faster than $r^2$, at sufficiently high $r$ the bulk energy gain will overtake the surface energy cost. This can be easily calculated by setting the derivative to zero, and we find the critical droplet radius
\begin{equation}
r^* = \frac{2\gamma}{\Delta\mu}.
\end{equation}
This results in an energy barrier of 
\begin{equation}
%	\Delta G^* = 
\Delta G (r^*) = \frac{16 \pi \gamma^3}{3 \Delta\mu^2}.
\end{equation}

\end{document}

