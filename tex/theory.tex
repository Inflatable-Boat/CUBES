% Credit: TeXniCie A-eskwadraat

\documentclass[thesis]{subfiles}

\begin{document}

\section{Theory}

Before we dive into the simulations, we'll first discuss a simple approach to crystal nucleation, and then how we make a simulation which gives us nucleation in a reasonable time frame, using umbrella sampling.

\subsection{Classical Nucleation Theory}

The simplest model of nucleation is well known, and goes by the name of classical nucleation theory (CNT). The theory applies when we have a system in a \emph{metastable} state. Heuristically it's as follows. The entire system is in state $A$, but under current conditions the free energy of the system would be lower in state $B$. However, by changing a droplet from state $A$ to state $B$, we create an interface between state $A$ and state $B$. Due to surface tension, this costs energy. Naturally, if this energy cost is greater than the energy gain from changing the state, then the system will, in order to minimize the free energy, make the droplet smaller to decrease surface tension at cost of the energy gain.

\end{document}

