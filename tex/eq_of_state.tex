% Credit: TeXniCie A-eskwadraat

\documentclass[thesis]{subfiles}

\begin{document}

\section{Equation of state}

One way to describe the properties of the system we're studying is by finding an \emph{equation of state}. This is an equation that relates the thermodynamic variables to one another. A simple example would be the ideal gas law, $\beta p = \rho$, where $\beta = 1/k_bT$ is the inverse temperature and $\rho = N/V$ is the density of the system. By running multiple $NpT$ ensembles for differing pressures $p$, and finding the equilibrium density $\rho$ for each $p$, we can find the equation of state. This has been done before \cite{van2017phase}, and we will compare our results to previously done research to see if the simulation is working as intended. \bigbreak

Two runs have been done per pressure: a compression run, starting at a relatively low density of 0.4 (i.e. in the fluid phase), and a melting run, starting in a relatively high density of 0.6, starting on a simple cubic lattice. The results are shown in figure \ref{fig:eos}. 

\begin{figure}
	\centering
	\includegraphics[width=0.8\linewidth]{images/v21_eq_of_state2}
	\caption{placeholder}
	\label{fig:eos}
\end{figure}

\end{document}

