% Credit: TeXniCie A-eskwadraat
%

% This is a subfile! Type your chapters/section, appendices, etc in here! This code can be run by itself to try for bugs or typo's, but you have to declare your master file for this to work.

% Declare the documentclass. Don't change \documentclass[...]{subfiles}, but replace the ... with the name of your master file.
\documentclass[thesis]{subfiles}

% Write the body of the chapter/section/other. You CAN NOT define preamble stuff in here (like \newcommand or \usepackage), that should be done in the preamble.tex instead (which will automatically apply to all subfiles).
\begin{document}

\newpage

\section{Measurements}

\subsection{Radial Distribution Function} \label{subsec:res_rdf}

We begin our study of the structure of the fluid by examining the radial distribution function in fluid phases for cubes for a range of densities from about $\rho\sigma^3 = 0.35$ to $\rho\sigma^3 = 0.45$. We choose these densities because they are below the liquid-crystal coexistence densities of cubes. The radial distribution functions can be seen in Figure \ref{fig:gofr_90liq}. We see that there are no cubes within a distance of $\sigma$, which is as expected, because we do not allow cubes to overlap. Furthermore, we see the shell structure typical of a liquid, as the density is high at around $r=1.4\sigma$, then low at $r=2.1\sigma$, then high again at $r=2.8\sigma$ etc.
The first minimum after the peak signifies the end of the first shell. We also see that as the density increases, the intensity of the second and third shell increase, and the shells become slightly thinner.

\begin{figure}[h]
	\centering
	\begin{subfigure}{0.49\textwidth}
		\includegraphics[width=\linewidth]{images/gofr_a157p3}
		\caption{$\beta p \sigma^3 = 3$, $\rho \sigma^3 = 0.35$. }
	\end{subfigure}
	\begin{subfigure}{0.49\textwidth}
		\includegraphics[width=\linewidth]{images/gofr_a157p4}
		\caption{$\beta p \sigma^3 = 4$, $\rho \sigma^3 = 0.39$. }
	\end{subfigure}
\\
\vspace{10pt}
	\begin{subfigure}{0.49\textwidth}
		\includegraphics[width=\linewidth]{images/gofr_a157p5}
		\caption{$\beta p \sigma^3 = 5$, $\rho \sigma^3 = 0.42$. }
	\end{subfigure}
	\begin{subfigure}{0.49\textwidth}
		\includegraphics[width=\linewidth]{images/gofr_a157p6}
		\caption{$\beta p \sigma^3 = 6$, $\rho \sigma^3 = 0.44$. }\label{fig:gofr_90liqp6}
	\end{subfigure}

	\caption{The $g(r)$ of systems of $N = 1728$ cubes over a range of pressures resulting in a range of densities from $\rho\sigma^3 = 0.35$ to $\rho\sigma^3 = 0.44$. Note the minimum is marked in every plot.}\label{fig:gofr_90liq}
\end{figure}

To see how much the particle shape influences the first shell, we now examine the $g(r)$ for a range of slant angle from $\phi = 84$\degr\ to $\phi = 66$\degr. The distribution functions can be found in Figure \ref{fig:gofr_non90liq}. As we can see, the distribution function changes only slightly. As the slant increases, the first minimum moves to slightly lower $r$, from $2.1\sigma$ at $\phi = 90$\degr (Figure \ref{fig:gofr_90liqp6}) to $2.0\sigma$ at $\phi = 66$\degr\ (see Figure \ref{fig:gofr_non90liqa66}). Furthermore, the intensity of the shells decreases, as the maximum of the $g(r)$ is almost 2 for cubes (see Figure \ref{fig:gofr_90liqp6}), but only 1.7 for cubes with a slant angle of $\phi = 66$\degr. It is important to note that the density also decreases, even though the pressure remains the same. This decrease in density could also partly explain the lower peaks in the density profile.

\begin{figure}[H]
	\centering
	\begin{subfigure}{0.49\textwidth}
		\includegraphics[width=\linewidth]{images/gofr_a147p6}
		\caption{Slant angle $\phi = 84^\circ$, $\beta p \sigma^3 = 6$, $\rho \sigma^3 = 0.44$. }
	\end{subfigure}
	\begin{subfigure}{0.49\textwidth}
		\includegraphics[width=\linewidth]{images/gofr_a137p6}
		\caption{Slant angle $\phi = 78^\circ$, $\beta p \sigma^3 = 6$, $\rho \sigma^3 = 0.43$. }
	\end{subfigure}
	\\
	\vspace{10pt}
	\begin{subfigure}{0.49\textwidth}
		\includegraphics[width=\linewidth]{images/gofr_a127p6}
		\caption{Slant angle $\phi = 72^\circ$, $\beta p \sigma^3 = 6$, $\rho \sigma^3 = 0.42$. }
	\end{subfigure}
	\begin{subfigure}{0.49\textwidth}
		\includegraphics[width=\linewidth]{images/gofr_a116p6}
		\caption{Slant angle $\phi = 66^\circ$, $\beta p \sigma^3 = 6$, $\rho \sigma^3 = 0.42$. }\label{fig:gofr_non90liqa66}
	\end{subfigure}
	
	\caption{The $g(r)$ of systems of $N = 1728$ slanted cubes over a range of slant angles from $66^\circ$ to $84^\circ$. Note the minimum is marked in every plot.}\label{fig:gofr_non90liq}
\end{figure}

The first neighbour shell of the fluid and crystal is expected to vary somewhat. Hence, we now explore the $g(r)$ for the crystal for cubes with a slant angle from $\phi = 66$\degr\ to $\phi = 90$\degr\ (cubes). See Figure \ref{fig:gofr_Xtal} for the radial distribution functions found for these slant angles. As seen in Figure \ref{fig:gofr_Xtal}, the crystal $g(r)$ exhibits a number of clear peaks, that correspond roughly to the distances between lattice sites in the simple cubic lattice\footnote{Note that except for the first peak, all other peaks are combinations of peaks from different lattice site distances.}. Moreover, the first peak is much narrower, and there is a clear distinction between nearest neighbours and second and third nearest neighbours, unlike in the liquid case. We see that as the slant increases, this distinction diminishes, as the first and second nearest neighbour distributions overlap much more. From these plots, we can infer that most first neighbours are within a distance of $1.4\sigma$, regardless of slant angle, although it must be noted that the density for each slant angle is slightly different. To find the distance of the first neighbour shell of cubes at the highest density of a liquid ($\rho \sigma^3 = 0.45$)\cite{van2017phase}, we use that the lattice constant scales as $a \sim \rho^{-1/3}$. Then we find that in the liquid of cubes, the first neighbours should be within a distance of $1.4\sigma (0.45/0.58)^{-1/3} \approx 1.52\sigma$. This result will be used in the next Section.

\begin{figure}[H]
	\centering
	\begin{subfigure}{0.49\textwidth}
	\includegraphics[width=\linewidth]{images/gofr_a157p10}
	\caption{Slant angle $\phi = 90^\circ$, $\beta p \sigma^3 = 10$, $\rho \sigma^3 = 0.58$. }
	\end{subfigure}
	\begin{subfigure}{0.49\textwidth}
		\includegraphics[width=\linewidth]{images/gofr_a147p10}
		\caption{Slant angle $\phi = 84^\circ$, $\beta p \sigma^3 = 10$, $\rho \sigma^3 = 0.56$. }
	\end{subfigure}
\\\vspace{10pt}
	\begin{subfigure}{0.49\textwidth}
		\includegraphics[width=\linewidth]{images/gofr_a137p10}
		\caption{Slant angle $\phi = 78^\circ$, $\beta p \sigma^3 = 10$, $\rho \sigma^3 = 0.54$. }
	\end{subfigure}
	\begin{subfigure}{0.49\textwidth}
		\includegraphics[width=\linewidth]{images/gofr_a127p12}
		\caption{Slant angle $\phi = 72^\circ$, $\beta p \sigma^3 = 12$, $\rho \sigma^3 = 0.54$. }
	\end{subfigure}
\\\vspace{10pt}
	\begin{subfigure}{0.49\textwidth}
		\includegraphics[width=\linewidth]{images/gofr_a116p15}
		\caption{Slant angle $\phi = 66^\circ$, $\beta p \sigma^3 = 15$, $\rho \sigma^3 = 0.54$. }
	\end{subfigure}
	
	\caption{The $g(r)$ of systems of $N = 1728$ slanted cubes over a range of slant angles from $\phi = 66$\degr\ to $\phi = 90$\degr\ at higher pressure. Note the minimum is marked in every plot. Also note that for smaller slant angles, the pressure was increased to keep the density somewhat similar.}\label{fig:gofr_Xtal}
\end{figure}

\subsection{Order Parameter} \label{subsec:order}

In this Section we will use the order parameters $q_4$ and $i_4$ to differentiate the crystal phase from the liquid state. In particular, in Section \ref{subsec:res_order cutoff} we examine the order parameter distributions on a global level to determine the order cutoff used for classifying cubes as ordered. In Section \ref{subsec:orientation vectors}, we address the ambiguity of the definition of orientational order for slanted cubes.

\subsubsection{Global order distributions}\label{subsec:res_order cutoff}

In Section \ref{subsec:res_rdf} we determined that the first neighbours in a simple cubic lattice at a density of 0.45 should be within a distance of $1.52\sigma$. Because this value is close to our preliminary value of $1.55\sigma$ that we used, the result justifies this value and we therefore call all cubes within a distance of $1.55\sigma$ \emph{bonded}.
In order to determine a proper cutoff for what bonds we consider to be \emph{ordered}, we set up an $NpT$ simulation with $N = 8000$ cubes which starts in a low density disordered liquid and set the pressure very high to `crush' the system to a high density crystal. When the system reaches densities of 0.40 to 0.55 in steps of 0.01, we record the $d_{q_4}$ and $d_{i_4}$ order distributions, both the total distribution as well as the distribution of order averaged per cube. These can be found in Figure \ref{fig:compression_cube}. Typical snapshots from both liquid and crystal configurations can be Figure \ref{fig:compression_cube_snapshots}.

\begin{figure}[h]
	\centering
	\begin{subfigure}{0.48\textwidth}
		\centering
		\includegraphics[width=0.80\linewidth,height=10\baselineskip]{images/compression_cube_q_}
		\caption{Distribution of translational order correlation.}\label{fig:compression_cubea}
	\end{subfigure}
	\begin{subfigure}{0.51\textwidth}
		\centering
		\includegraphics[width=\linewidth,height=10\baselineskip]{images/compression_cube_qbar}
		\caption{Distribution of translational order correlation averaged per cube.}\label{fig:compression_cubeb}
	\end{subfigure}\\\vspace{10pt}
	\begin{subfigure}{0.48\textwidth}
		\centering
		\includegraphics[width=0.80\linewidth,height=10\baselineskip]{images/compression_cube_i_}
		\caption{Distribution orientational order correlation.}\label{fig:compression_cubec}
	\end{subfigure}
	\begin{subfigure}{0.51\textwidth}
		\centering
		\includegraphics[width=\linewidth,height=10\baselineskip]{images/compression_cube_ibar}
		\caption{Distribution orientational order correlation averaged per cube.}\label{fig:compression_cubed}
	\end{subfigure}
	\caption{The translational as well as orientational order distribution ranging from a low density fluid to a high density crystal. %See figure \ref{fig:compression_cube_snapshots} for snapshots of the low density and high densities.
	An $NpT$ simulation of $N = 8000$ cubes starting at a density of $\rho \sigma^3 = 0.1$ was run at a very high pressure of $\beta p \sigma^3 = 10000$. When the density reaches $\rho \sigma^3 = 0.4, 0.41$ et cetera, the translational and orientational order is calculated for all bonds. On the left the total distribution is plotted, on the right the distribution of the average order for per cube is plotted.}
	\label{fig:compression_cube}
\end{figure}

In Figure \ref{fig:compression_cube} we can see a clear difference between the distribution of a low density liquid and a high density crystal. In Figures \ref{fig:compression_cubea} and \ref{fig:compression_cubec}, we can see that as the density increases, and more of the system becomes crystalline, and the distribution shifts from a broad range centered around $d_{q_4} = 0.3$ and $d_{i_4} = 0.1$ to a narrow distribution around $\overline{d_{q_4}}, \overline{d_{i_4}} = 0.85.$ The distinction between these distributions is even more obvious in Figures \ref{fig:compression_cubeb} and \ref{fig:compression_cubed}, where the correlations have been averaged per cube. These last results are especially powerful for discerning crystalline cubes from liquid ones. This can be explained by the fact that in a liquid, order correlations are distributed more or less randomly. So, if we average a number of bonds, we are more likely to end up with a number closer to the center of the distribution.
Interesting to note is that though the liquid is mostly disordered, there is a slight net positive amount of order in the liquid phase, which is more pronounced in the translational order. Also noteworthy is that for both the translational and orientational order, even for the order averaged per cube, the distribution seems to shift continuously from low order to higher order, rather than seeing the `liquid' distribution shrinking while the 'crystal' distribution grows. This is likely due to cubes in the interface between crystal and the liquid, which have an intermediate amount of order.

From Figures \ref{fig:compression_cubeb} and \ref{fig:compression_cubed} we can infer that in the liquid phase (densities $\rho \sigma^3 < 0.45$), almost no cubes have an average translational bond correlation with their neighbours of more than 0.6, and no average orientational bond correlation of more than 0.5. With these results we can set the bond order parameter cutoff to 0.6. Any correlation larger than 0.6 will be considered an ordered bond.

\begin{figure}[h]
	{\centering
		\hfill
		\begin{subfigure}{0.42\textwidth}
			\centering
			\includegraphics[width=0.95\linewidth]{images/compression_cube_pf040}
			\caption{The system at a density of $\rho \sigma^3 = 0.40$.}
		\end{subfigure}\hfill
		\begin{subfigure}{0.42\textwidth}
			\centering
			\includegraphics[width=0.95\linewidth]{images/compression_cube_pf055}
			\caption{The system at a density of $\rho \sigma^3 = 0.55$.}
		\end{subfigure}
		\hfill}
	\caption{Two snapshots of the system consisting of 8000 cubes, one at low density resulting in a fluid, and one at high density resulting in a crystal.}
	\label{fig:compression_cube_snapshots}
\end{figure}

%\newpage

\subsubsection{Orientational Order Parameter for Slanted Cubes}\label{subsec:orientation vectors}

In Section \ref{subsubsec:orient order param}, we mentioned an ambiguity in defining the orientational order axes used to calculate the orientational order in slanted cubes. For a given cube, we defined three axes, namely the normal vectors on each face of the cube. These coincide with the edges of the cube. For cubes with a slant angle of less than $\phi = 90$\degr\ however, these two definitions will give different set of vectors. Actually, the one set is just a rotated version of the other, which should give us the same results because the order parameter we use is rotationally invariant \cite{steinhardt1983bond}. Alternatively, we can consider the normal vectors defined by the normal vectors on the faces of the `unslanted cube.' That is, we consider the slanted cube to be not slanted. Given the symmetries of the slanted cube, there are actually two ways to do `unslant' the cube, so by picking one we break the symmetry of the slanted cube\footnote{One could argue that therefore this is a flawed choice. We will nevertheless explore this option.}. \\
We will now investigate how the choices of these vectors affect the ability to measure orientational order. In order to do so, we do the same simulation as in the previous Section, but with slanted cubes with a slant angle of $\phi = 72$\degr. That is, we ran an $NpT$ simulation with $N = 8000$ slanted cubes which start in a low density disordered liquid and set the pressure very high to `crush' the system to a high density crystal. When the system reaches densities of $\rho \sigma^3 = 0.40$ to $\rho \sigma^3 = 0.55$ in steps of 0.01, we record the $i_4$ order distributions, calculated using the three different sets of orientation vectors described above. The results can be found in Figure \ref{fig:orient_comparison}.
From these figures we can see all of the methods give us very similar results. Though we expected the methods using slanted face normals (Fig. \ref{fig:orient_comparison_sl1} and \ref{fig:orient_comparison_sl2}) and edges of the slanted cube (Fig. \ref{fig:orient_comparison_e1} and \ref{fig:orient_comparison_e2}) to be similar because the vectors they define are a simply a rotated version of each other, we can also see that using unslanted edges (Fig. \ref{fig:orient_comparison_unsl1} and \ref{fig:orient_comparison_unsl2}) has near identical results. We thus conclude that the difference between calculation orientational order from the different sets of three vectors does not matter.

\begin{figure}[H]
	\centering
	\begin{subfigure}{0.48\textwidth}
		\centering
		\includegraphics[width=0.80\linewidth,height=10\baselineskip]{images/compression_72_unsl_i_.pdf}
		\caption{$d_{i_4}$ distribution using unslanted face normals.}\label{fig:orient_comparison_unsl1}
	\end{subfigure}
	\begin{subfigure}{0.51\textwidth}
		\centering
		\includegraphics[width=\linewidth,height=10\baselineskip]{images/compression_72_unsl_ibar.pdf}
		\caption{$\overline{d_{i_4}}$ distribution using unslanted face normals.}\label{fig:orient_comparison_unsl2}
	\end{subfigure}\\\vspace{10pt}
	\begin{subfigure}{0.48\textwidth}
		\centering
		\includegraphics[width=0.80\linewidth,height=10\baselineskip]{images/compression_72_sl_i_.pdf}
		\caption{$d_{i_4}$ distribution using using slanted face normals.}\label{fig:orient_comparison_sl1}
	\end{subfigure}
	\begin{subfigure}{0.51\textwidth}
		\centering
		\includegraphics[width=\linewidth,height=10\baselineskip]{images/compression_72_sl_ibar.pdf}
		\caption{$\overline{d_{i_4}}$ distribution using slanted face normals.}\label{fig:orient_comparison_sl2}
	\end{subfigure}\\\vspace{10pt}
	\begin{subfigure}{0.48\textwidth}
		\centering
		\includegraphics[width=0.80\linewidth,height=10\baselineskip]{images/compression_72_e_i_.pdf}
		\caption{$d_{i_4}$ distribution using edges.}\label{fig:orient_comparison_e1}
	\end{subfigure}
	\begin{subfigure}{0.51\textwidth}
		\centering
		\includegraphics[width=\linewidth,height=10\baselineskip]{images/compression_72_e_ibar.pdf}
		\caption{$\overline{d_{i_4}}$ distribution using edges.}\label{fig:orient_comparison_e2}
	\end{subfigure}
	\caption{The orientational order distribution of a system of $N = 8000$ slanted cubes with a slant angle of 72\degr\ over a density from 0.4 to 0.55. In (a) and (b) the order was calculated using the unslanted cube face normals, in (c) and (d) using the slanted cube face normals, and , in (e) and (f) using the edges of the slanted cubes. Note that the scale on the vertical axis is not the same in each plot.}
	\label{fig:orient_comparison}
\end{figure}

%\begin{figure}[H]
%	\centering
%	\begin{subfigure}{0.48\textwidth}
%		\centering
%		\includegraphics[width=\linewidth]{images/clus1}
%		\caption{A sample snapshot of a 1728 cubes in a liquid phase, with some ordered clusters.}
%	\end{subfigure}\hfill
%	\begin{subfigure}{0.48\textwidth}
%		\centering
%		\includegraphics[width=\linewidth]{images/clus3}
%		\caption{The same system, but at a higher density most cubes belong to a single cluster.}
%	\end{subfigure}
%	\caption{Two examples of how our system of interest is visualized. Disordered cubes are drawn small so we can `see through them,' and ordered cubes are drawn at true size, and colored depending on which cluster they belong to.}
%	\label{fig:sample_snapshot}
%\end{figure}

\end{document}







































